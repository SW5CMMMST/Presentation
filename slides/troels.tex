\section{Implementation}
    \subsection{Automated Code Generation}
    \begin{frame}[t]{Code Generation From Model}\framesubtitle{Hello World!}
        Ideally
        \begin{enumerate}
            \item Make model,
            \item Verify it,
            \item ??? (Automated Code Generation),
            \item Implementation is over.
        \end{enumerate}
        \bigskip
        \begin{itemize}
            \item <2->Assuming the 3rd step is correct then it will work, because the model works!
            \item <3->However this is a \textcolor{orange}{difficult} problem, and code generation will be done manually. 
        \end{itemize}
    \end{frame}
    \subsection{Implementing for the Arduino}
    \begin{frame}[t]{Implementing for the Arduino}\framesubtitle{Platform and toolchain}
        \begin{itemize}
            \item C/C++-like language,
            \item $< 500$ LOC,
            \item Uses a Library (RadioHead) to communicate,
            \item Relatively simple to make given the model,
            \item Hardest problem being synchronization related,
            \item Uses 8560 bytes out of 32256 bytes of storage (26 \%),
            \item Uses 672 bytes of 2048 bytes of memory (32 \%).
        \end{itemize}

    \end{frame}
\section{Demonstration}
    \begin{frame}[t]{Demonstration}
        \begin{itemize}
            \item 4 devices,
            \item Usercode!
            \item Can start multiple at the same time,
            \item Green LED on while sending,
            \item Red LED on while receiving,
            \item Multiple buttons turning on LEDs/actuators.
        \end{itemize}
    \end{frame}
